\documentclass[bachelor, och, pract, times]{SCWorks}
\usepackage[T2A]{fontenc}
\usepackage[utf8]{inputenc}
\usepackage{graphicx}
\usepackage[sort,compress]{cite}
\usepackage{amsmath}
\usepackage{amssymb}
\usepackage{amsthm}
\usepackage{fancyvrb}
\usepackage{longtable}
\usepackage{array}
\usepackage[english,russian]{babel}
\usepackage{tabularx}
\usepackage{svg}
\usepackage{multirow}
\usepackage{verbatim}
\usepackage{import}
\usepackage{xifthen}
\usepackage{pdfpages}
\usepackage{transparent}
\usepackage[pdfborder={0 0 0}]{hyperref}
\begin{document}


% ==============================================================================

\section{Алгоритмы имитационной модели}

В начале проведения симуляции управление находится у ведущей программы $\pi_0$. В ходе её выполнения управление может передаваться программным процессам $\pi_a$, $\pi_{bs}$ и $\pi_{es}$. Алгоритмы данных процессов приведены ниже.  

Алгоритм программного процесса $\pi_a$ — \textbf{«Поступление требования в систему»} — включает следующие действия:
    \begin{enumerate}
        \item Определение момента $t = t_{now} + t_a$ поступления требования в систему, где $t_{now}$ — текущий момент модельного времени.
        \item Выход из процесса $\pi_a$ и возврат к ведущей программе.
        \item Формирование требования и разбиение его на фрагменты. Этот процесс заключается в определении числа фрагментов, что является случайной величиной, создании для каждого фрагмента элемента коллекции, который будет содержать набор переменных, хранящих такие атрибуты фрагмента, как момент времени его появления в системе, идентификатор требования, к которому он относится, и его текущее состояние.
        \begin{enumerate}
            \item Первому атрибуту присваивается значение текущего момента модельного времени $t_{now}$.
        \end{enumerate}
        \begin{enumerate}
            \item Последнему атрибуту присваивается значение $-1$, означающее, что этот фрагмент находится в очереди.
        \end{enumerate}
        \item Все фрагменты требования ставятся в очередь системы.
        \item Переход на шаг 1.
    \end{enumerate}

Алгоритм программного процесса $\pi_{bs}$ — \textbf{«Начало обслуживания»} —включает следующие действия:
    \begin{enumerate}
        \item Определение номера свободного прибора обслуживания.
        \item Изменение состояние этого прибора на значение «занят».
        \item Атрибуту «текущее состояние» взятого из очереди фрагменту требования присваивается значение, равное номеру прибора.
        \item  Определение момента $t = t_{now} + t_{es}$ завершения обслуживания фрагмента требования, где $t_{now}$ — текущий момент модельного времени, а $t_{es}$ — длительность обслуживания данного фрагмента.
        \item Выход из процесса $\pi_{bs}$ и возврат к ведущей программе.
\end{enumerate}

Алгоритм программного процесса $\pi_{es}$ — \textbf{«Завершение обслуживания»} — включает следующие действия:
    \begin{enumerate}
        \item Изменение состояния завершившего обслуживание прибора на значение «свободен».
        \item Атрибуту «текущее состояние» обслуженного фрагмента требования присваивается значение, равное общему числу приборов, что означает его готовность к сборке.
        \item Проверка: если все фрагменты данного требования были обслужены и ожидают сборки, то они покидают систему как единое требование.
        \item Если очередь системы не пуста, то перейти к программному процессу $\pi_{bs}$ с шага 1. В противном случае, выход из процесса $\pi_{es}$ и возврат к ведущей программе.
\end{enumerate}

Алгоритм \textbf{ведущей программы} $\pi_0$:
    \begin{enumerate}
        \item Определение начальных условий. 
        \item Если очередь системы не пуста и хотя бы один прибор обслуживания свободен, то передать управление программному процессу $\pi_{bs}$ с шага 1. Иначе выполнить шаг 3.
        \item Из таблицы расписания событий выбрать событие с минимальным моментом активации.
        \item Продвинуть текущий момент модельного времени $t_{now}$ до момента активации выбранного события.
        \item Выполняется проверка условия $t_{now} \leq t_{max}$, где $t_{now}$ — общее время моделирования. Если неравенство не выполняется, то процесс моделирования завершается, и вычисляются характеристики модели. Конец алгоритма. В противном случае переход к шагу 6.
        \item В соответствии с событием выбирается программный процесс, которому необходимо передать управление:
        \begin{enumerate}
            \item Если выбранным событием является генерация нового требования, то управление получает процесс $\pi_{a}$.
            \item Если выбранным событием является завершение обслуживания фрагмента требования, то управление получает процесс $\pi_{es}$.
        \end{enumerate}
    \end{enumerate}
    
\section{Вероятностно-временные характеристики имитационной модели}

В ходе осуществления функционирования имитационной модели ведущая программа ведёт сбор статистических данных и вычисляет оценки некоторых характеристик системы. 

Записывается суммарное время $t_n$, $n=0,1,2,\dots$, нахождения в системе ровно $n$ требований. Затем с помощью выражения
\[\hat{p}_n = \frac{t_n}{t_{max}}\] 
вычисляются оценки стационарных вероятностей состояний системы.

Кроме того, после окончания обслуживания каждого требования, его общая длительность $\tau$ пребывания в системе записывается в суммирующую переменную, содержащую аналогичную характеристику для всех обслуженных требований. С помощью выражения 
\[\hat{u} = \frac{\sum \limits^Q_{i = 1} \tau_i}{Q},\]
где $Q$ — количество обслуженных требований, вычисляется оценка среднего времени пребывания требования в системе.

\section{Структура программы имитационной модели}

По описанным ранее алгоритмам была разработана программа имитационной модели. Для реализации использовался язык программирования Python и модули \textit{math}, \textit{numpy} и \textit{json}. Программа позволяет вычислять оценки следующих вероятностно-временных характеристик рассматриваемой системы массового обслуживания типа $M^{[x]}/M/C$ при заданных параметрах:
\begin{itemize}
    \item стационарное распределение вероятностей состояний системы $\hat{p}$.
    \item среднее времени прибывания требования $\hat{u}$ в системе.
\end{itemize}


Программа состоит из двух модулей:
\begin{enumerate}
\item System.py — модуль, в котором содержится класс \textit{Mx\_M\_C}, описывающий соответствующую систему. В классе определены следующие атрибуты:
\begin{itemize}
    \item \textit{lambda\_} — интенсивность входящего потока;
    \item \textit{servers\_count} — число обслуживающих приборов в системе;
    \item \textit{mu} — интенсивность обслуживания на приборах системы;
    \item \textit{servers\_states} — вектор состояния обслуживающих приборов системы;
    \item  \textit{demands} — коллекция, содержащая требования, находящиеся в очереди системы;
    \item \textit{last\_state} — номер предыдущего состояния, то есть числа требований, системы.
\end{itemize}
В классе \textit{Mx\_M\_C} содержатся следующие методы:
\begin{itemize}
    \item \textit{arrival\_time} — возвращает момент времени, когда очередной требование появится в системе;
    \item \textit{service\_time} — возвращает момент времени, когда система завершит обслуживание требования;
    \item \textit{pack\_size} — возвращает число фрагментов, на которые разобьётся требование;
    \item \textit{export\_states} — сохраняет данные о времени пребывания системы в каждом состоянии в соответствующей файл;
    \item \textit{export\_demands} — сохраняет данные о фрагментах требований в системе в соответствующей файл;
    \item \textit{import\_states} — загружает данные о времени пребывания системы в каждом состоянии из соответствующего файла;
    \item \textit{import\_demands} — загружает данные о фрагментах тербований в системе из соответствующего файла;
    \item \textit{current\_demands} — возвращает идентификаторы присутствующих в системе требований;
    \item \textit{update\_time\_states} — обновляет данные о времени пребывания системы в каждом состоянии.
\end{itemize}

\item main.py — основной модуль, содержащий точку входа в программу. Содержит исходные данные для моделирования:
\begin{itemize}
    \item \textit{t\_max} — общее время моделирования;
    \item {lambda\_} — интенсивность входящего потока;
    \item  \textit{mu} — интенсивность обслуживания приборами системы;
    \item \textit{servers\_count} — число обслуживающих приборов;
    \item \textit{b} — среднее число фрагментов, на которые разбивается требование;
    \item \textit{t} — текущее значение модельного времени;
    \item \textit{indicator} — индикатор, отражающий, произошло ли на данном моменте модельного времени какое-либо событие;
    \item \textit{schedule} — таблица временных отметок активации событий;
    \item \textit{ready\_packs\_count} — количество обслуженных требований;
    \item \textit{sum\_packs\_life\_time} — суммарное время нахождения требований в системе;
\end{itemize}
В теле модуля содержится функция \textit{simulation}, которая осуществляет процесс симуляции, продвигая модельное время вперёд и обрабатывая возникающие события, а также собирает данные статистики.
\end{enumerate}


\section{Результаты имитационного моделирования}

В ходе проведения экспериментов с реализованной имитационной моделью с целью нахождения оценок характеристик рассматриваемой системы массового обслуживания типа $M^{[x]}$ были получены примеры результатов моделирования.

На основе данных экспериментов можно сделать вывод, что разработанная модель применима для анализа системы массового обслуживания с делением и слиянием требований: интенсивность поступления требований в систему, интенсивностью их обслуживания, количеством обслуживающих приборов и средним числом фрагментов, на которые разделяется требование.

\textbf{Пример 1.}

Рассматривается система массового обслуживания $M^{[x]}/M/C$.

Параметры генерации требований следующие: длительности интервалов между поступающими требованиями имеют экспоненциальное распределение, интенсивность поступления $\lambda = 1/10$.

Параметры системы определены следующим образом: 

\begin{itemize}
    \item длительность обслуживания имеет экспоненциальное распределение, 
    \item интенсивность обслуживания $\mu 1/281$, 
    \item число обслуживающих приборов $С = 150$,
    \item среднее число фрагментов разбиения требования $\overline{b} = 5$.
\end{itemize}

\textbf{Результаты моделирования системы массового обслуживания:}
\begin{itemize}
    \item оценка среднего времени пребывания требования в системе $\hat{u} = $,
    \item оценка вероятностей стационарного распределения состояний системы $\hat{p} = ()$.
\end{itemize}

\textbf{Пример 2.}

Рассматривается система массового обслуживания $M^{[x]}/M/C$.

Параметры генерации требований следующие: длительности интервалов между поступающими требованиями имеют экспоненциальное распределение, интенсивность поступления $\lambda = 1/10$.

Параметры системы определены следующим образом: 

\begin{itemize}
    \item длительность обслуживания имеет экспоненциальное распределение, 
    \item интенсивность обслуживания $\mu 1/281$, 
    \item число обслуживающих приборов $С = 150$,
    \item среднее число фрагментов разбиения требования $\overline{b} = 5$.
\end{itemize}

\textbf{Результаты моделирования системы массового обслуживания:}
\begin{itemize}
    \item оценка среднего времени пребывания требования в системе $\hat{u} = $,
    \item оценка вероятностей стационарного распределения состояний системы $\hat{p} = ()$.
\end{itemize}

\end{document}
